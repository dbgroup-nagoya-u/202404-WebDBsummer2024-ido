% !TEX root = main.tex

\section{はじめに}

Mass木は\Bptree{}にトライ木構造を適応させることで,キャッシュ効率を改善させた.
\Bcforest{}では\Bctree{}に対し,同様の改善を図る.

Mass木の観点では\Bptree{}から\Bctree{}になることにより,ロックフリーによる書き込み性能の改善を図る.
また\Bctree{}の観点では,整数型や文字列型などのバイナリ比較可能なキーに対し,キャッシュ効率の改善を図る.

\section{関連研究}
\label{sec:relatedwork}
関連の深い索引構造として,同時実行制御においてロックを取得しない\Bctree{},および\Bptree{}にトライ木の構造を組み合わせたMass木について紹介する.

\subsection{\Bctree{}}
\Bctree{}はマッピングテーブル,ノード内バッファという構造上の特徴とCAS命令を用いたロックフリー索引である.
\Bctree{}の概形を\Fig{\ref{fig:bc_tree-structure}}に示す.

\begin{figure}[t]
    \centering
    \includegraphics{./figures/Bc-structure.pdf}
    \caption{\Bctree{}の概形}
    \label{fig:bc_tree-structure}
\end{figure}

\subsubsection{データ構造の概観}
\Bptree{}と同様に,\Bctree{}は索引層およびデータ層によって構成される.
索引層のノード(中間ノード)は分割キーと子ノードへのポインタの組を格納し,木の下方への検索を補助する.
構造は\Blinktree{}に則っており,各ノードが同じ階層の右兄弟への参照リンクを持つ.
ノード間の繋がりはマッピングテーブルにより仮想化する.
各ノードは自身の子ノードや兄弟ノードへのポインタを直接持つ代わりにマッピングテーブル上のID(logical page ID, LPID)を持つ.
各ノードへの参照はマッピングテーブルを用いた間接参照を採用し,マッピングテーブル内の物理ポインタを差し替えることでそのノードへの参照を一括で変更する.

各ノードの領域は不変領域と可変領域(ノード内バッファ)に分けられる.
不変領域はノードヘッダおよびソート済みのレコードを格納する.
ヘッダは不変領域の情報を管理し,構造変更時のみその値が変更される.
可変領域はステータスワードの格納と差分レコードを挿入するための書き込みバッファの役割を果たす.
ステータスワードは可変領域の情報を管理し,ノードの現在の状態や残容量などを管理する.

\subsubsection{レコード操作の概観}
ステータスワードをCAS命令で更新することによって,ロックフリーな書き込みを実現している.
各ノードへ構造変更操作を行う際は,構造変更後のノードから構造変更前のノードへ物理リンクを張り,古いノードへの参照を可能にする.

\subsection{Mass木}
Mass木は\Bptree{}を基本単位とした階層構造やレコードメタデータの削除により,キャッシュ効率を改善した索引構造である.
Mass木の概形を\Fig{\ref{fig:masstree}}に示す.

\begin{figure}[t]
    \centering
    \includegraphics{./figures/masstree.pdf}
    \caption{Mass木の概形}
    \label{fig:masstree}
\end{figure}

Mass木は複数の\Bptree{}とlayer構造から構成される.
Layer~0はキーの先頭0~7~byteで構成される\Bptree{}である.
先頭8~byteで一意性が確保できる場合には,Layer~0で完結する.
先頭8~byteで一意性が確保出来ない場合,Layer~1(キーの8~15~byteで構成される\Bptree{})を作成し,Layer~0からLayer~1への物理リンクを張る.
同様にして,複数の\Bptree{}やLayerを作成し,トライ木に似た構造を持つのがMass木の特徴である.
Mass木は整数型や文字列型など,分割が可能なキーに限定することで上記に示す階層化(共通部分の集約)を可能にしている.

また,Mass木は固定長キーおよび固定長ペイロードに特化したノードレイアウトを利用している.
\Bctree{}のような可変長キーおよび可変長ペイロードに対応する索引構造では,各レコードに対応する固定長レコードメタデータを利用することでノード内のレコードの配置等を管理している.
Mass木では,8~byte分割によりキーが8~byteで固定される.
更に,ペイロードに関してもポインタの活用やインラインを固定長に限定することで,メタデータの利用を回避している.

以上のトライ木構造の利用やレコードメタデータの削除により,Mass木はキャッシュ効率を改善している.

\section{\Bcforest{}の構造}
\label{sec:bc_forest_structure}

\Bcforest{}はBc木をバイナリ比較可能なキーに最適化した索引構造である.
Mass木のように8~byte単位でキーを分割,階層分けし,各階層でのレコード管理にはBc木を利用する.
つまり,各Bc木の中間ノードでは固定長の部分キーおよび子ノードへのポインタのみを管理することとなり,レコードメタデータの除外によるキャッシュ効率の改善が可能となる.
一方で,葉ノードではposting listを用いて共通する部分キーを持つレコードを管理し,少数のレコードのみからなる下層の生成を抑制する.

\subsection{中間ノードにおけるレコードメタデータの除外}
\Bcforest{}では中間ノードはキーを8~byteで分割しているため,固定長キーとして扱うことが出来る.
また,ペイロードは子ノードへのポインタであるため固定長である.
この特性を利用し,\Bcforest{}内の\Bctree{}における中間ノード内のレコードメタデータの除外を行う.
これにより,索引層における探索性能の改善及びキャッシュ効率を改善を図る.

\subsection{葉ノードにおけるposting listの導入}

\begin{figure*}[t]
    \centering
    \includegraphics{./figures/memory.pdf}
    \caption{Mass木の空間利用率}
    \label{fig:memory}
\end{figure*}

Mass木においては,空間利用効率が問題となる.
\Fig{\ref{fig:memory}}は末尾3~byteが異なる19~byteの2~キーを格納したMass木の索引構造である.
先頭8~byteが共通するため,layer~1を作成する.
同様に8~16~byte目が共通するため,layer~2を作成する.
Mass木では本来,1~つのノードに最大16~個のレコードを格納することが出来る.
しかしlayer~1では,1~つのレコードしか格納されていない状態でlayer~2を作成している.
layer~1にのみ注目すると,空間利用率は約6.25~\%しかない.

\begin{figure}[t]
    \centering
    \includegraphics{./figures/posting_list.pdf}
    \caption{posting listの導入}
    \label{fig:posting_list}
\end{figure}

\Bcforest{}では空間利用率の改善として,posting listを導入する.
posting listでは,1~つのキーに対して複数のペイロードを対応付けることが出来る.
\Fig{\ref{fig:posting_list}}はposting listを導入した際の索引構造を示したものである.
layer~1においてposting list作成することで,layer~2の無駄な階層化と空間利用率の悪化を防ぐ.


\section{\Bcforest{}のノード操作}
\label{sec:node_operation}

\subsection{書き込み}

\subsection{読み取り}

\section{\Bcforest{}の構造変更操作}
\label{sec:smo}

\subsection{統合}
\subsection{分割}
\subsection{新層作成}

\section{おわりに}
\label{sec:conclusion}

