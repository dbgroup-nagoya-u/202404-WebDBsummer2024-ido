% !TEX root = main.tex

\section{はじめに}

\section{関連研究}
\label{sec:relatedwork}

\subsection{\Bctree{}}
\subsection{Mass木}

\section{\Bcforest{}の構造}
\label{sec:bc_forest_structure}

\Bcforest{}は\Bctree{}を基本単位とする階層を持つ索引構造である.
\Bcforest{}の概形を  に示す.
\Sec{\ref{sec:bc_forest_structure}}では\Bctree{}との差異である,トライ木構造の適応とposting listの導入について説明する.

\subsection{トライ木構造の適応}

Mass木は\Bptree{}にトライ木構造を適応させることで,キャッシュ効率を改善させた.
\Bcforest{}では\Bctree{}に対し,同様の改善を図る.

Mass木の観点では\Bptree{}から\Bctree{}になることにより,ロックフリーによる書き込み性能の改善を図る.
また\Bctree{}の観点では,整数型や文字列型などのバイナリ比較可能なキーに対し,キャッシュ効率の改善を図る.

\subsection{posting listの導入}

Mass木においては,空間利用効率が問題となる.
・  は末尾3~byteが異なる19~byteの2~キーを格納したMass木の索引構造である.
先頭8~文字が共通するため,layer1を作成する.
同様に8~16~byte目が共通するため,layer2を作成する.
Mass木では本来,1~つのノードに最大16~個のレコードを格納することが出来る.
しかしlayer1では,1~つのレコードしか格納されていない状態でlayer2を作成している.
layer1にのみ注目すると,空間利用率は約6.25~\%しかない.

\Bcforest{}では空間利用率の改善として,posting listを導入する.
posting listでは,1~つのキーに対して複数のペイロードを対応付けることが出来る.
・  はposting listを導入した際の索引構造を示したものである.
layer1においてposting list作成することで,layer2の無駄な階層化と空間利用率の悪化を防ぐ.


\section{\Bcforest{}のノード操作}
\label{sec:node_operation}

\subsection{書き込み}

\subsection{読み取り}

\section{\Bcforest{}の構造変更操作}
\label{sec:smo}

\subsection{統合}
\subsection{分割}
\subsection{新層作成}

\section{おわりに}
\label{sec:conclusion}

