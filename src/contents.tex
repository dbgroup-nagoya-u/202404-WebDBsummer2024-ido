% !TEX root = main.tex

\section{はじめに}

\section{関連研究}
\label{sec:relatedwork}

\subsection{\Bctree{}}
\subsection{Mass木}

\section{\Bcforest{}の構造}
\label{sec:bc_forest_structure}

\Bcforest{}は\Bctree{}を基本単位とする階層を持つ索引構造である.
\Bcforest{}の概形をに示す.
\Sec{\ref{sec:bc_forest_structure}}では\Bctree{}との差異である,posting listの導入とトライ木構造の適応について説明する.

\subsection{posting listの導入}

Mass木において,

\subsection{トライ木構造の適応}

\section{\Bcforest{}のノード操作}
\label{sec:node_operation}

\subsection{書き込み}
\subsection{読み取り}

\section{\Bcforest{}の構造変更操作}
\label{sec:smo}

\subsection{統合}
\subsection{分割}
\subsection{新層作成}

\section{おわりに}
\label{sec:conclusion}

