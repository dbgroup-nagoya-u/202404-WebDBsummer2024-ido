% !TEX root = main.tex

\section{はじめに}
ムーアの法則の終焉により,CPUのコア単体性能は限界に達しつつある.
一方で,IT技術の発展に伴い管理すべきデータは爆発的に増えつつある.
この状況に対処するため,現在のコンピュータ技術は複数のコアを用いて処理を行うマルチスレッド処理が主流である.
データベース分野においても例外ではなく,近年メニーコアなどを前提としたインメモリデータベースの研究が進んでいる.
データベースの構成要素の1つである索引技術も同様に,メニーコア・大容量メモリに適合させる必要がある.

代表的な索引構造である\Bptree{}~\cite{book:dbsystem}では,ロックを用いた同時実行制御が行われている.
しかし,マルチスレッド処理においてロックによる同時実行制御は多数の待ちスレッドが発生するため,スケーラビリティが悪化する.
そこで,\Bptree{}をロックフリー化させた索引としてBw木~\cite{book:Bwtree}やBz木~\cite{book:Bztree},著者らの研究室で開発しているロックフリーB+木(\Bctree{})が提案されている.

また,インターネットの普及に伴いURLの管理やECサイトにおける文字列検索など,特定のキーに対し効率的に処理する索引構造が求められている.
Mass木~\cite{book:Masstree}は,文字列型や整数型などのバイナリ比較可能なキーに特化した索引構造の1~つである.
\Bptree{}を階層的に作成することにより,キャッシュ効率を改善している.

本研究の\Bcforest{}では,著者らの研究室で開発しているロックフリーB+木(\Bctree{})に対し,Mass木のと同様の拡張および性能改善を行う.
特に,本論文ではその構造および操作について述べる.

本稿の構成は以下の通りである.
\Sec{\ref{sec:relatedwork}}では,関連研究としてロックフリー索引やバイナリ比較可能なキーに対し最適化した索引について概説する.
次に,\Sec{\ref{sec:bc_forest_structure}}で\Bcforest{}の構造について説明し,\Sec{\ref{sec:node_operation}}および\Sec{\ref{sec:smo}}で\Bcforest{}の操作について述べる.
最後に,\Sec{\ref{sec:conclusion}}で本稿のまとめと今後の方針を述べる.

\section{関連研究}
\label{sec:relatedwork}

\subsection{\Bctree{}}
\subsection{Masstree}

\section{\Bcforest{}の構造}
\label{sec:bc_forest_structure}

\section{\Bcforest{}のノード操作}
\label{sec:node_operation}

\subsection{書き込み}
\subsection{読み取り}

\section{\Bcforest{}の構造変更操作}
\label{sec:smo}

\subsection{統合}
\subsection{分割}
\subsection{新層作成}

\section{おわりに}
\label{sec:conclusion}
本稿では\Bctree{}にMass木と同様のトライ木構造を適応させた\Bcforest{}について提案し,その構造および操作を紹介した.
今後は提案した索引構造を実装するとともに,Mass木や近年提案されているBw木やBz木といったロックフリー索引との性能を比較検証する.